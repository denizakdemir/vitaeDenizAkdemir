%% start of file `template.tex'.
%% Copyright 2006-2015 Xavier Danaux (xdanaux@gmail.com).
%
% Adapted to be an Rmarkdown template by Mitchell O'Hara-Wild
% 8 February 2019
%
% This work may be distributed and/or modified under the
% conditions of the LaTeX Project Public License version 1.3c,
% available at http://www.latex-project.org/lppl/.


\documentclass[11pt,a4paper,]{moderncv}

% moderncv themes
\moderncvstyle{casual}                             % style options are 'casual' (default), 'classic', 'banking', 'oldstyle' and 'fancy'

\definecolor{color0}{rgb}{0,0,0}% black
\definecolor{color1}{HTML}{3873B3}% custom
\definecolor{color2}{rgb}{0.45,0.45,0.45}% dark grey

\usepackage[scaled=0.86]{DejaVuSansMono}

\providecommand{\tightlist}{%
	\setlength{\itemsep}{0pt}\setlength{\parskip}{0pt}}
\def\donothing#1{#1}
\def\emaillink#1{#1}

%\nopagenumbers{}                                  % uncomment to suppress automatic page numbering for CVs longer than one page

% character encoding
%\usepackage[utf8]{inputenc}                       % if you are not using xelatex ou lualatex, replace by the encoding you are using
%\usepackage{CJKutf8}                              % if you need to use CJK to typeset your resume in Chinese, Japanese or Korean

% adjust the page margins
\usepackage[scale=0.75,footskip=60pt]{geometry}
%\setlength{\hintscolumnwidth}{3cm}                % if you want to change the width of the column with the dates
%\setlength{\makecvheadnamewidth}{10cm}            % for the 'classic' style, if you want to force the width allocated to your name and avoid line breaks. be careful though, the length is normally calculated to avoid any overlap with your personal info; use this at your own typographical risks...



% personal data
\name{}{Dr.~Deniz Akdemir}
\title{Senior Clinical Data Scientist - Statistical Genomics Consultant}
\address{Bowling Green, Ohio, USA}{}{}

\phone[mobile]{+1 607 262 6875} % Phone number
\email{\donothing{\href{mailto:deniz.akdemir.work@gmail.com}{\nolinkurl{deniz.akdemir.work@gmail.com}}}}
\homepage{denizakdemir.github.io} % Personal website
\social[linkedin]{deniz-akdemir-50735314a}

\social[github]{denizakdemir}




% \extrainfo{additional information}                 % optional, remove / comment the line if not wanted




% Pandoc CSL macros

%----------------------------------------------------------------------------------
%            content
%----------------------------------------------------------------------------------
\begin{document}
%\begin{CJK*}{UTF8}{gbsn}                          % to typeset your resume in Chinese using CJK
%-----       resume       ---------------------------------------------------------
\makecvtitle



\section{Professional Profile}\label{professional-profile}

Deniz Akdemir is an AI/ML specialist with extensive experience in
machine learning, data analytics, and computational intelligence. His
work at the intersection of statistical methodology and machine learning
has produced groundbreaking tools like ml4t2e, which bridges traditional
survival analysis and modern machine learning for time-to-event data.
His GitHub repositories demonstrate practical implementation of machine
learning techniques through Python packages like trainselpy for optimal
training set selection. At the National Marrow Donor Program, he applies
sophisticated AI/ML approaches to stem cell transplant data, developing
predictive models for donor optimization that directly impact patient
outcomes.

\section{Core Competencies}\label{core-competencies}

\begin{itemize}
\tightlist
\item
  Machine Learning \& Deep Learning
\item
  Data Analytics and Visualization
\item
  Algorithm Development
\item
  Statistical Modeling
\item
  Research \& Innovation
\end{itemize}

\section{Professional Experience}\label{professional-experience}

\begin{itemize}
\tightlist
\item
  \textbf{Senior Clinical Data Scientist}, National Marrow Donor Program
  (2023 - Current)\\
  Leading machine learning initiatives for donor-recipient matching
  optimization using advanced predictive models and deep learning
  approaches for clinical outcome prediction.
\item
  \textbf{Clinical Data Scientist}, National Marrow Donor Program (2021
  - 2023)\\
  Applied statistical and machine learning methods to stem cell
  transplant data, developing novel algorithms for patient outcome
  prediction and donor selection optimization.
\item
  \textbf{Open Source ML Developer}, GitHub (2019 - Present)\\
  Developed ml4t2e (Machine Learning for Time-to-Event) package that
  bridges traditional survival analysis with modern machine learning
  approaches and trainselpy for optimizing training populations using
  Python.
\item
  \textbf{ML \& AI Researcher}, University College Dublin (2019 -
  2021)\\
  Researched machine learning applications for heterogeneous data
  integration, developing novel algorithms for combining genomic and
  phenotypic datasets.
\end{itemize}

\section{Education}\label{education}

\begin{itemize}
\tightlist
\item
  \textbf{PhD. in Statistics}, Bowling Green State University (2009)
\item
  \textbf{M.A.~in Applied Statistics}, Bowling Green State University
  (2004)
\item
  \textbf{M.S. in Statistics}, Middle East Technical University (2003)
\item
  \textbf{B.A. in Business Administration}, Middle East Technical
  University, Ankara, Turkey, 1999.
\end{itemize}

\section{Research Interests}\label{research-interests}

Deniz's research focuses on integrating statistical methods with machine
learning to create robust and interpretable AI models. His GitHub
repositories, including ml4t2e, demonstrate his work on bridging
traditional survival analysis with modern machine learning approaches.
He is particularly interested in applying AI techniques to clinical
data, developing predictive models for patient outcomes, and creating
optimization algorithms for complex biological systems. His current work
explores deep learning architectures for time-to-event analysis and
reinforcement learning for decision support in healthcare settings.

\section{Collaboration Network}\label{collaboration-network}

\pandocbounded{\includegraphics[keepaspectratio]{Vitae_AI_ML_files/figure-latex/coauthor_network-1.png}}

\section{Core Skills}\label{core-skills}

\begin{itemize}
\tightlist
\item
  \textbf{Machine Learning \& AI:} Deep learning architectures, neural
  networks, reinforcement learning, natural language processing,
  computer vision, predictive modeling, feature engineering, ensemble
  methods.
\item
  \textbf{Statistical Methods:} Multivariate analysis, Bayesian methods,
  time series analysis, high-dimensional data modeling, anomaly
  detection, classification algorithms.
\item
  \textbf{Data Science:} Data mining, exploratory data analysis, A/B
  testing, experimental design, causal inference, predictive modeling,
  data visualization.
\item
  \textbf{Programming \& Tools:} Expert in R and Python (NumPy, Pandas,
  PyTorch, TensorFlow, scikit-learn), proficient in SAS and C++, SQL,
  cloud computing (AWS, Azure), big data technologies (Spark).
\item
  \textbf{Applied Research:} Clinical data analysis, genomic data
  integration, algorithm development, optimization techniques, model
  deployment, interpretability methods.
\end{itemize}

\section{Career Summary}\label{career-summary}

\begin{itemize}
\tightlist
\item
  \textbf{Senior Clinical Data Scientist, National Marrow Donor Program,
  Minneapolis, USA (2023 - Current):} Engaged in statistical and machine
  learning analysis of stem cell transplant data, focusing on research
  into donor optimization. Applied for grants and submitted manuscripts
  to peer-reviewed journals. Wrote patent applications.
\item
  \textbf{Clinical Data Scientist, National Marrow Donor Program,
  Minneapolis, USA (2021 - 2023):} Engaged in statistical and machine
  learning analysis of stem cell transplant data, focusing on research
  into donor optimization.
\item
  \textbf{Postdoctoral Research Associate, School of Agriculture and
  Food Science, University College Dublin, Dublin, Ireland (2019 -
  2021):} Conducted research on methods for combining heterogeneous
  genomic and phenotypic datasets and prepared statistical software for
  data analysis.
\item
  \textbf{Statistical Consultant, Cornell Statistical Consulting Unit,
  Cornell University, Ithaca, NY, USA (2017 - 2019):} Provided
  statistical consulting services for researchers at Cornell University,
  including the preparation and presentation of statistics workshops.
\item
  \textbf{Postdoctoral Research Associate, Department of Plant Breeding
  and Genetics, Cornell University, Ithaca, NY, USA (2011 - 2017):}
  Focused on research developing new methodologies in genomic selection
  and prediction, mixed models, and machine learning, advising graduate
  students and preparing statistical software.
\item
  \textbf{Visiting Assistant Professor, Department of Statistics and
  Actuarial Science, University of Central Florida, Orlando, FL, USA
  (2010 - 2011):} Responsibilities included teaching Data Mining
  Methodology, Theoretical Statistics, Applied Time Series Analysis, and
  Nonparametric Statistics.
\item
  \textbf{Visiting Assistant Professor, Department of Mathematics and
  Statistics, Ohio Northern University, Ada, OH, USA (2009 - 2010):}
  Taught Statistics for Professionals, Statistics for Engineers, and
  Statistical Computing, catering to various undergraduate levels.
\end{itemize}

\section{Key AI/ML Projects}\label{key-aiml-projects}

\begin{itemize}
\item
  \textbf{ml4t2e (Machine Learning for Time-to-Event):} Developed an
  open-source framework that bridges traditional survival analysis with
  modern machine learning approaches, enabling more accurate prediction
  of time-to-event outcomes in clinical settings. The package implements
  novel algorithms that outperform traditional Cox regression models.
\item
  \textbf{Deep Learning for Donor-Recipient Matching:} Leading a neural
  network approach to optimize stem cell transplant outcomes by modeling
  complex interactions between donor and recipient genetic and clinical
  variables, significantly improving prediction accuracy compared to
  traditional methods.
\item
  \textbf{Reinforcement Learning for Clinical Decision Support:}
  Implementing reinforcement learning algorithms to optimize sequential
  treatment decisions in healthcare settings, with applications in
  personalized medicine and treatment planning optimization.
\item
  \textbf{trainselpy:} Developed a Python implementation of training
  population optimization algorithms, employing advanced machine
  learning techniques including genetic algorithms and simulated
  annealing to improve predictive model performance through strategic
  data selection.
\end{itemize}

\section{Workshops and Training
Sessions}\label{workshops-and-training-sessions}

\begin{itemize}
\tightlist
\item
  \textbf{Machine Learning in Healthcare Workshop:} Conducted
  specialized training sessions on applying machine learning to clinical
  data, focusing on practical implementation and model interpretability.
\item
  \textbf{AI Applications in Genomics:} Presented at various
  international conferences including EFI conference in Geneva and ASHI
  Annual Meeting in San Antonio, demonstrating AI applications in
  genomic data analysis.
\end{itemize}

\section{Selected Publications (Machine Learning
Focus)}\label{selected-publications-machine-learning-focus}

\begin{enumerate}
\def\labelenumi{\arabic{enumi}.}
\item
  \textbf{Akdemir, D.}, Isidro-Sánchez, J., \& Jannink, J. L. (2016).
  Genome-wide prediction models that incorporate de novo GWAS results.
  \emph{PLoS ONE}, 11(8), e0161054. (346 citations)
\item
  \textbf{Akdemir, D.}, Jannink, J. L., \& Isidro-Sánchez, J. (2015).
  Optimization of genomic selection training populations with a genetic
  algorithm. \emph{Genetics Selection Evolution}, 47(1), 38. (196
  citations)
\item
  \textbf{Akdemir, D.} \& Isidro-Sánchez, J. (2019). Multi-objective
  optimized genomic breeding strategies for sustainable food
  improvement. \emph{Heredity}, 122(5), 672-683. (103 citations)
\item
  \textbf{Akdemir, D.} \& Beavis, W. D. (2021). TrainSel: an R package
  for selection of training populations. \emph{BMC Bioinformatics},
  22(1), 1-10. (33 citations)
\item
  \textbf{Akdemir, D.} \& Isidro-Sánchez, J. (2021). Training set
  optimization for sparse phenotyping in genomic selection. \emph{G3:
  Genes, Genomes, Genetics}, 11(10), jkab249. (32 citations)
\item
  \textbf{Akdemir, D.}, Isidro-Sánchez, J., \& Leyer, M. (2020).
  Multi-omics approaches for genomic selection in plant breeding
  programs. \emph{Journal of Experimental Botany}, 71(18), 5215-5226.
\end{enumerate}

\textbf{Software Development Publications:}

\begin{enumerate}
\def\labelenumi{\arabic{enumi}.}
\item
  \textbf{Akdemir, D.} (2022). ml4t2e: Machine Learning for
  Time-to-Event. An R package that bridges traditional survival analysis
  with modern machine learning approaches.
  \url{https://github.com/denizakdemir/ml4t2e}
\item
  \textbf{Akdemir, D.} (2022). trainselpy: A pure Python implementation
  of the TrainSel R package for optimal selection of training
  populations. \url{https://github.com/denizakdemir/trainselpy}
\end{enumerate}

\section{Professional References}\label{professional-references}

\begin{itemize}
\tightlist
\item
  \textbf{Dr.~Yung-Tsi Bolon}

  \begin{itemize}
  \tightlist
  \item
    \textbf{Affiliation:} Director, Immunobiology \& Bioinformatics
    Research, NMDP, Minneapolis, Minnesota, United States
  \item
    \textbf{Relationship:} Supervisor at the National Marrow Donor
    Program
  \item
    \textbf{Contact:}
    \href{mailto:ybolon@nmdp.org}{\nolinkurl{ybolon@nmdp.org}}
  \end{itemize}
\item
  \textbf{Dr.~Julio Isidro-Sanchez}

  \begin{itemize}
  \tightlist
  \item
    \textbf{Affiliation:} Associate Professor: Centro de Biotecnologia y
    Genomica de Plantas, Universidad Politecnica de Madrid, Instituto
    Nacional de Investigacion y Tecnologia Agraria y Alimentaria, Campus
    de Montegancedo - UPM, 28223-Pozuelo de Alarcon, Madrid, Spain
  \item
    \textbf{Relationship:} Expert in plant breeding and genetics,
    collaborator on various projects
  \item
    \textbf{Contact:}
    \href{mailto:j.isidro@upm.es}{\nolinkurl{j.isidro@upm.es}}
  \end{itemize}
\item
  \textbf{Dr.~Jhonathan Pedroso}

  \begin{itemize}
  \tightlist
  \item
    \textbf{Affiliation:} Research Scientist at Corteva Agriscience,
    Corteva, Johnston, Iowa, USA
  \item
    \textbf{Relationship:} Industry partner in genomic tool development,
    contributed to software enhancements
  \item
    \textbf{Contact:}
    \href{mailto:jhowpd@gmail.com}{\nolinkurl{jhowpd@gmail.com}}
  \end{itemize}
\item
  \textbf{Dr.~Lynn Johnson}

  \begin{itemize}
  \tightlist
  \item
    \textbf{Affiliation:} Interim Director and Statistical Consultant,
    Cornell Statistical Consulting Unit, Cornell University, Ithaca, NY,
    USA
  \item
    \textbf{Relationship:} Coworker at the Cornell Statistical
    Consulting Unit
  \item
    \textbf{Contact:}
    \href{mailto:lms86@cornell.edu}{\nolinkurl{lms86@cornell.edu}}
  \end{itemize}
\item
  \textbf{Dr.~Roberto Fritsche Neto}

  \begin{itemize}
  \tightlist
  \item
    \textbf{Affiliation:} Assistant Professor, Department of Plant,
    Environmental Management \& Soil Sciences, LSU
  \item
    \textbf{Relationship:} Collaborator on various projects
  \item
    \textbf{Contact:}
    \href{mailto:rneto1@lsu.edu}{\nolinkurl{rneto1@lsu.edu}}
  \end{itemize}
\end{itemize}


\end{document}

%\clearpage\end{CJK*}                              % if you are typesetting your resume in Chinese using CJK; the \clearpage is required for fancyhdr to work correctly with CJK, though it kills the page numbering by making \lastpage undefined
\end{document}


%% end of file `template.tex'.
